% Options for packages loaded elsewhere
\PassOptionsToPackage{unicode}{hyperref}
\PassOptionsToPackage{hyphens}{url}
\PassOptionsToPackage{dvipsnames,svgnames,x11names}{xcolor}
%
\documentclass[
  letterpaper,
  DIV=11,
  numbers=noendperiod]{scrartcl}

\usepackage{amsmath,amssymb}
\usepackage{iftex}
\ifPDFTeX
  \usepackage[T1]{fontenc}
  \usepackage[utf8]{inputenc}
  \usepackage{textcomp} % provide euro and other symbols
\else % if luatex or xetex
  \usepackage{unicode-math}
  \defaultfontfeatures{Scale=MatchLowercase}
  \defaultfontfeatures[\rmfamily]{Ligatures=TeX,Scale=1}
\fi
\usepackage{lmodern}
\ifPDFTeX\else  
    % xetex/luatex font selection
\fi
% Use upquote if available, for straight quotes in verbatim environments
\IfFileExists{upquote.sty}{\usepackage{upquote}}{}
\IfFileExists{microtype.sty}{% use microtype if available
  \usepackage[]{microtype}
  \UseMicrotypeSet[protrusion]{basicmath} % disable protrusion for tt fonts
}{}
\makeatletter
\@ifundefined{KOMAClassName}{% if non-KOMA class
  \IfFileExists{parskip.sty}{%
    \usepackage{parskip}
  }{% else
    \setlength{\parindent}{0pt}
    \setlength{\parskip}{6pt plus 2pt minus 1pt}}
}{% if KOMA class
  \KOMAoptions{parskip=half}}
\makeatother
\usepackage{xcolor}
\setlength{\emergencystretch}{3em} % prevent overfull lines
\setcounter{secnumdepth}{-\maxdimen} % remove section numbering
% Make \paragraph and \subparagraph free-standing
\ifx\paragraph\undefined\else
  \let\oldparagraph\paragraph
  \renewcommand{\paragraph}[1]{\oldparagraph{#1}\mbox{}}
\fi
\ifx\subparagraph\undefined\else
  \let\oldsubparagraph\subparagraph
  \renewcommand{\subparagraph}[1]{\oldsubparagraph{#1}\mbox{}}
\fi


\providecommand{\tightlist}{%
  \setlength{\itemsep}{0pt}\setlength{\parskip}{0pt}}\usepackage{longtable,booktabs,array}
\usepackage{calc} % for calculating minipage widths
% Correct order of tables after \paragraph or \subparagraph
\usepackage{etoolbox}
\makeatletter
\patchcmd\longtable{\par}{\if@noskipsec\mbox{}\fi\par}{}{}
\makeatother
% Allow footnotes in longtable head/foot
\IfFileExists{footnotehyper.sty}{\usepackage{footnotehyper}}{\usepackage{footnote}}
\makesavenoteenv{longtable}
\usepackage{graphicx}
\makeatletter
\def\maxwidth{\ifdim\Gin@nat@width>\linewidth\linewidth\else\Gin@nat@width\fi}
\def\maxheight{\ifdim\Gin@nat@height>\textheight\textheight\else\Gin@nat@height\fi}
\makeatother
% Scale images if necessary, so that they will not overflow the page
% margins by default, and it is still possible to overwrite the defaults
% using explicit options in \includegraphics[width, height, ...]{}
\setkeys{Gin}{width=\maxwidth,height=\maxheight,keepaspectratio}
% Set default figure placement to htbp
\makeatletter
\def\fps@figure{htbp}
\makeatother
\newlength{\cslhangindent}
\setlength{\cslhangindent}{1.5em}
\newlength{\csllabelwidth}
\setlength{\csllabelwidth}{3em}
\newlength{\cslentryspacingunit} % times entry-spacing
\setlength{\cslentryspacingunit}{\parskip}
\newenvironment{CSLReferences}[2] % #1 hanging-ident, #2 entry spacing
 {% don't indent paragraphs
  \setlength{\parindent}{0pt}
  % turn on hanging indent if param 1 is 1
  \ifodd #1
  \let\oldpar\par
  \def\par{\hangindent=\cslhangindent\oldpar}
  \fi
  % set entry spacing
  \setlength{\parskip}{#2\cslentryspacingunit}
 }%
 {}
\usepackage{calc}
\newcommand{\CSLBlock}[1]{#1\hfill\break}
\newcommand{\CSLLeftMargin}[1]{\parbox[t]{\csllabelwidth}{#1}}
\newcommand{\CSLRightInline}[1]{\parbox[t]{\linewidth - \csllabelwidth}{#1}\break}
\newcommand{\CSLIndent}[1]{\hspace{\cslhangindent}#1}

\usepackage{algorithm2e}
\usepackage{bbm}
\newcommand{\qiq}{\qquad \implies \qquad}
\newcommand{\qiffq}{\qquad \iff \qquad}
\newcommand{\qaq}{\qquad \text{and} \qquad}
\newcommand{\qoq}{\qquad \text{or} \qquad}
\newcommand{\settf}{\text{ \emph{:} }}
\KOMAoption{captions}{tableheading}
\makeatletter
\makeatother
\makeatletter
\makeatother
\makeatletter
\@ifpackageloaded{caption}{}{\usepackage{caption}}
\AtBeginDocument{%
\ifdefined\contentsname
  \renewcommand*\contentsname{Table of contents}
\else
  \newcommand\contentsname{Table of contents}
\fi
\ifdefined\listfigurename
  \renewcommand*\listfigurename{List of Figures}
\else
  \newcommand\listfigurename{List of Figures}
\fi
\ifdefined\listtablename
  \renewcommand*\listtablename{List of Tables}
\else
  \newcommand\listtablename{List of Tables}
\fi
\ifdefined\figurename
  \renewcommand*\figurename{Figure}
\else
  \newcommand\figurename{Figure}
\fi
\ifdefined\tablename
  \renewcommand*\tablename{Table}
\else
  \newcommand\tablename{Table}
\fi
}
\@ifpackageloaded{float}{}{\usepackage{float}}
\floatstyle{ruled}
\@ifundefined{c@chapter}{\newfloat{codelisting}{h}{lop}}{\newfloat{codelisting}{h}{lop}[chapter]}
\floatname{codelisting}{Listing}
\newcommand*\listoflistings{\listof{codelisting}{List of Listings}}
\makeatother
\makeatletter
\@ifpackageloaded{caption}{}{\usepackage{caption}}
\@ifpackageloaded{subcaption}{}{\usepackage{subcaption}}
\makeatother
\makeatletter
\@ifpackageloaded{tcolorbox}{}{\usepackage[skins,breakable]{tcolorbox}}
\makeatother
\makeatletter
\@ifundefined{shadecolor}{\definecolor{shadecolor}{rgb}{.97, .97, .97}}
\makeatother
\makeatletter
\makeatother
\makeatletter
\makeatother
\ifLuaTeX
  \usepackage{selnolig}  % disable illegal ligatures
\fi
\IfFileExists{bookmark.sty}{\usepackage{bookmark}}{\usepackage{hyperref}}
\IfFileExists{xurl.sty}{\usepackage{xurl}}{} % add URL line breaks if available
\urlstyle{same} % disable monospaced font for URLs
\hypersetup{
  colorlinks=true,
  linkcolor={blue},
  filecolor={Maroon},
  citecolor={Blue},
  urlcolor={Blue},
  pdfcreator={LaTeX via pandoc}}

\author{}
\date{}

\begin{document}
\ifdefined\Shaded\renewenvironment{Shaded}{\begin{tcolorbox}[sharp corners, interior hidden, breakable, enhanced, frame hidden, boxrule=0pt, borderline west={3pt}{0pt}{shadecolor}]}{\end{tcolorbox}}\fi

\hypertarget{model-setup}{%
\subsection{Model Setup}\label{model-setup}}

\hypertarget{demographics}{%
\subsubsection{Demographics}\label{demographics}}

\begin{itemize}
\tightlist
\item
  There are two locations \(j\in\{1,2\}\).
\item
  A continuum of workers is characterized by their abilities, which are
  denoted as \(x \in \mathcal{X}\).

  \begin{itemize}
  \tightlist
  \item
    The total measure of workers is normalized to 1.
  \item
    Workers types follow an exogenous distribution \(\ell(x)\).
  \item
    Within each location, there is an endogenous distribution of workers
    denoted as \(\ell^{j}(x)\).
  \item
    Let the total population at location \(j=1\) be \(\mu\) and at
    location \(j=2\) be \(1-\mu\).
  \end{itemize}
\item
  A continuum of firms is characterized by their technology, which is
  denoted as \(y \in \mathcal{Y}\).

  \begin{itemize}
  \tightlist
  \item
    The total measure of firms is normalized to 1.
  \item
    Firms follow an exogenous distribution \(\Phi\).
  \end{itemize}
\end{itemize}

\hypertarget{technology}{%
\subsubsection{Technology}\label{technology}}

\begin{itemize}
\tightlist
\item
  There is an exogenous cost associated with posting \(v\) job
  opportunities in location \(j\), denoted as \(c_{j}(v)\), where
  \(c_{j}(v) \geq 0\).

  \begin{itemize}
  \tightlist
  \item
    I assume this cost function is increasing, convex, and potentially
    dependent on the location.
  \end{itemize}
\item
  Both workers and firms discount the future at the rate \(\beta\).
\item
  Workers have the ability to move across locations:

  \begin{itemize}
  \tightlist
  \item
    Workers select a mixed strategy for their job search, denoted as
    \(\phi^j_i(x)\), where
    \(\phi^j_i(x) = \{\phi^j_i(x, j')\}_{j' \in \mathcal{J}}\)
    represents the probability that a worker of type \(x\) from location
    \(j\) searches in location \(j'\), and \(i \in \{u,s\}\) refers to
    the employment status of the worker.
  \item
    Each search strategy incurs an associated cost denoted as
    \(c_s(\phi_i^j(x))\).
  \item
    When a worker moves from one location to another (from \(j\) to
    \(j'\)), they must pay a relocation cost denoted as
    \(F^{j \to j'} \geq 0\), with \(F^{j \to j} = 0\) for movements
    within the same location.
  \end{itemize}
\item
  Unemployed workers receive instant utility in each location denoted as
  \(b(x,j)\).\footnote{In
    (\protect\hyperlink{ref-liseMacrodynamicsSortingWorkers2017}{Lise
    and Robin 2017}), \(b(x)\) represents unemployment benefits.
    However, I adopt a more general approach to account for differences
    in the cost of living across locations.}
\item
  Firms have access to a production technology that operates at the
  match level and depends on the location, represented as \(f_j(x, y)\):
  \[f_1(x,y) = f(\Omega(x,\mu), y) \qaq  f_2(x,y) = f(\Omega(x, 1-\mu), y)\]

  \begin{itemize}
  \tightlist
  \item
    Where \(\Omega(x, \mu)\) captures agglomeration effects:
    \[\Omega(x, \mu) = x(1+ A(1-e^{\nu \mu})x)\]
  \end{itemize}
\end{itemize}

\hypertarget{job-search}{%
\subsubsection{Job Search}\label{job-search}}

Both unemployed and employed workers engage in job search activities.
Let's denote \(s\) as the search intensity of an employed worker,
\(s < 1\) is normalized such that an unemployed worker has a search
intensity of \(1\). To compute the total search intensity in location
\(j\), we need to consider the following expression:

\[
L^j = \sum_{j'\in\{1,2\}}\left[\int \phi_u^{j'}(x,j)u^{j'}(x) dx + s\int\int \phi_s^{j'}(x,j)h^{j'}(x,y)dx dy\right]
\]

Total search intensity (\(L^j\)) in location \(j\) accounts for the
search behaviors of both employed and unemployed workers from all other
locations (\(j'\)) in the model. It takes into account the probabilities
of workers from everywhere else searching for jobs in location \(j\),
capturing the spatial dynamics of the labor market.

Now, let's introduce some additional definitions:

\begin{itemize}
\item
  Let \(v^j(y)\) represent the number of job opportunities posted by a
  firm \(y\) in location \(j\).
\item
  \(V^j = \int v^j(y) dy\) represents the total number of job
  opportunities posted location \(j\).
\end{itemize}

Now, let's define \(M^j = M(L^j, V^j)\) as the number of job matches in
location \(j\). With these definitions in place, we can calculate the
following probabilities:

\begin{itemize}
\item
  The probability that an unemployed worker contacts a vacancy in
  location \(j\) is given by: \[p^j = \frac{M^j}{L^j}\] Here, \(sp^j\)
  represents the probability that an employed worker contacts a vacancy.
\item
  The probability that a firm contacts any searching worker is defined
  as: \[q^j = \frac{M^j}{V^j}\]
\item
  Additionally, let \(\theta^j = V^j / L^j\), represent the market
  tightness in location \(j'\).
\end{itemize}

Finally, a job offer is a draw of a firm productivity from the vacancy
distribution in each location \(\Gamma^j(\cdot)\) with pdf
\(\gamma^j(\cdot )\), note that \[\gamma^j(y) = p^j\frac{v^j(y)}{V^j}\]

\hypertarget{dynamic-programming-problem}{%
\subsection{Dynamic Programming
Problem}\label{dynamic-programming-problem}}

Consider the following notation:

\begin{itemize}
\tightlist
\item
  \(U^{j}(x)\) represents the value for an unemployed worker of type
  \(x\) in location \(j'\).
\item
  The value of receiving a job offer depends on the employment status of
  the worker:

  \begin{itemize}
  \tightlist
  \item
    \(W^{j\to j'}_{0}(x,y)\) denotes the value of a type-\(x\)
    unemployed worker in location \(j\) who is hired by a firm of type
    \(y\) in location \(j'\).
  \item
    \(W^{j\to j'}_{1}(x,y \to y')\) represents the value offered by a
    type \(y'\) firm in location \(j'\) to a type \(x\) worker currently
    employed at a type \(y\) firm in location \(j\).
  \end{itemize}
\item
  \(J^j(x,y)\) denotes the value of a match between a type \(x\) worker
  and a type \(y\) firm inlocation \(j\).
\end{itemize}

Now, let's define the joint surplus of a match between a type \(x\)
unemployed worker in location \(j\) and a type \(y\) firm in location
\(j'\) as:

\[S^{j\to j'}(x,y) = J^{j'}(x,y) - [U^{j}(x) + F^{j\to j'}]\]

\(S^{j\to j'}(x,y)\) represent much better off (or worse off) the worker
and the firm are when they are matched compared to when the worker is
unemployed in location \(j\) and considering the cost of moving from
\(j\) to \(j'\). If \(S^{j\to j'}(x,y)\) is positive, it indicates a
positive gain from the match; if it's negative, it represents a net
loss.

\hypertarget{search-strategy}{%
\subsubsection{Search Strategy}\label{search-strategy}}

In this section I describe the search strategy of workers. I specify
what do I mean by a search strategy and how do workers choose their
search strategy. I also describe the cost of search and how it is
related to the search strategy. Finally I derive the optimal search
strategy for each worker type and location. Throughout this section I
abstract from the employment status of the worker.

I assume that workers are \emph{rational inattentive} in the style of
(\protect\hyperlink{ref-simsImplicationsRationalInattention2003}{Christopher
A. Sims 2003}),
(\protect\hyperlink{ref-simsRationalInattentionLinearQuadratic2006}{Christopher
A. Sims 2006}) and
(\protect\hyperlink{ref-matejkaRationalInattentionDiscrete2015}{Matějka
and McKay 2015}). Workers have prior knoledge of the value of moving to
every location \(j'\) \(\{u^{j\to j'}(x)\}_{j \in \mathcal{J}}\), note
that this is dependent on the type of each worker and their location at
the moment of decition making. Assume that this prior knolege is
described by a joint distribution
\(G\left(\{u^{j\to j'}(x)\}_{j \in \mathcal{J}} \right)\).

To refine this knowledge, workers can aquire information about the value
of each location by searching. Workers are in escence acquiring
informatin to reduce uncertainty (i.e.~reduce entropy) associated with
their prior knoledge. I make the siplyfing assumption that
\(\phi^j_0(x)\), the unconditional distribution of of choosing each
locaiton is uniform, this is:

\[\phi^j_0(x, j') = \frac{1}{\mid \mathcal{J} \mid} \qquad 
\forall j' \in \mathcal{J}, x \in \mathcal{X}\]

Thus type \(x\) workers in location \(j\) are faced with the problem:

\begin{align*}
\max_{\phi_i^j(x)} \left( \sum_{j' \in \mathcal{J}} \phi_i^j(x, j') u^{j\to j'}(x) - c(\phi_i^j(x), \phi^j_0(x))  \right)\\
\text{s.t.} \quad \sum_{j'\in \mathcal{J}}\phi^j_i(x, j') = 1 \\
\quad \text{and} \quad \phi^j_i(x, j') \geq 0 \quad \forall j' \in \mathcal{J}
\end{align*}

With \(c(\phi_i^j(x), \phi^j_0(x))\) being the cost of reducing the
entropy of the prior (in this case the assumption is the prior is no
information). The cost is proportional to the Kullback--Leibler
divergence (also called relative entropy and \(I\)-divergence) between
the selected distribution and the prior (uniform). More information on
the Kullback--Leibler divergence
\href{https://en.wikipedia.org/wiki/Kullback\%E2\%80\%93Leibler_divergence}{here}.
For simplicity, denote \(c(\phi_i^j(x), \phi^j_0(x)) = c(\phi_i^j(x))\)

\begin{equation}
\label{eq-cost-search}
c_s(\phi^j_i(x)) = c \sum_{j'\in \mathcal{J}}\phi^j_i(x, j')\log{(J\phi^i_j(x, j'))}
\end{equation}

If processing information where costless i.e.~\(c = 0\) the worker would
be able to perfectly direct their search towards the highest value
location. With costly information procesing (\(c > 0\)), the worker can
still select to search randomly (i.e.~uniform distribution) and pay the
asociated cost of \(0\), when the worker starts directing their search
towards a particular location (the asusmption is that this requieres,
aquiring more information of one of the potential values relative to the
others) then the cost grows unboundedly large as the strategy gets
closer to perfectly directing search to a particular location.

\begin{itemize}
\tightlist
\item
  Besides
  (\protect\hyperlink{ref-matejkaRationalInattentionDiscrete2015}{Matějka
  and McKay 2015}) which formulates the problem in general terms, some
  works in the literature that use this cost structure are
  (\protect\hyperlink{ref-wuPartiallyDirectedSearch2020}{Wu 2020}) and
  (\protect\hyperlink{ref-cheremukhinTargetedSearchMatching2020}{Cheremukhin,
  Restrepo-Echavarria, and Tutino 2020}).
\end{itemize}

Solving the maximization problem we get the optimal strategy for each
worker as:

\begin{align*}
\max_{\phi_i^j(x)} \left( \sum_{j' \in \mathcal{J}} \phi_i^j(x, j') u^{j\to j'}(x) - c \sum_{j'\in \mathcal{J}}\phi^j_i(x, j')\log{(J\phi^i_j(x, j'))} \right) \\
\text{s.t.} \quad \sum_{j'\in \mathcal{J}}\phi^j_i(x, j') = 1 \\
\quad \text{and} \quad \phi^j_i(x, j') \geq 0 \quad \forall j' \in \mathcal{J}
\end{align*}

I'll ignore the non-negativity constraints and write the Lagrangean of
the problem:
\[\mathcal{L}(\phi^j(x), \lambda) =  \sum_{j'\in\mathcal{J}} \phi^j(x, j') u^{j\to j'}(x) - c \sum_{j'\in \mathcal{J}}\phi^j(x, j')\log{(J\phi^j(x, j'))} + \lambda \left(\sum_{j'\in \mathcal{J}}\phi^j(x, j') - 1\right)\]

First order conditions of the problem give us:

\begin{align*}
  [\phi^j(x, j')] &:\quad u^{j\to j'}(x) - c - c \log[J \phi^j(x, j')] = \lambda \\
  [\lambda] &: \quad \sum_{j'\in \mathcal{J}}\phi^j(x, j') = 1
\end{align*}

Take any two \(j_1\), \(j_2\) we have that
\[u^{j\to j_1}(x)  - c \log[J \phi^j(x, j_1)] =u^{j\to j_2}(x)  - c \log[J \phi^j(x, j_2)]\]

thus

\begin{align*}
  \frac{u^{j\to j_1}(x) - u^{j\to j_2}(x) }{c} &= \log\left(\frac{ \phi^j(x, j_1)}{ \phi^j(x, j_2)}\right) \\ 
  &\implies \quad \frac{ \phi^j(x, j_1)}{ \phi^j(x, j_2)} = \frac{
    e^{
       u^{j\to j_1}(x) / c 
        }
        }{e^{
u^{j\to j_2}(x) / c
        }
        }
\end{align*}

Fix any \(\hat{j}\), then we can write any other \(j'\in\mathcal{J}\) in
therms of \(\hat{j}\) and plug into the constraint to get:

\begin{align*}
  \sum_{j'\in \mathcal{J}}\phi^j(x, j') &= \sum_{j'\in \mathcal{J}}\frac{e^{u^{j\to j'}(x) }}{e^{u^{j\to \hat{j}}(x) }} \phi^j(x, \hat{j})\\  &= \frac{\phi^j(x, \hat{j})}{e^{u^{j\to \hat{j}}(x) / c}} \sum_{j\in \mathcal{J}}e^{u^{j\to j'}(x) / c} = 1
\end{align*}

Which we can solve to obtain

\begin{equation}
\label{eq:optimal-search-strategy}
    \phi^j(x,\hat{j}) = \frac{e^{u^{j\to \hat{j}}(x) / c}}{\sum_{ j' \in \mathcal{J}}e^{u^{j\to j'}(x) / c}}
\end{equation}

The non-negativity constraints are satisfied because the exponential
function is always positive.

\textbf{Note:} I could have follow
(\protect\hyperlink{ref-matejkaRationalInattentionDiscrete2015}{Matějka
and McKay 2015}) and have more complex prior beliefs, the result of this
would be that the optimal strategy would be biased by the prior
attractivenes of each location. In terms of closed form solution I would
get something like:
\[\phi^j(x,\hat{j}) = \frac{e^{(u^{j\to \hat{j}}(x) + \zeta^{j \to \hat{j}}(x) )/ c}}{\sum_{ j' \in \mathcal{J}}e^{(u^{j\to j'}(x) + \zeta^{j \to \hat{j}}(x)) / c}}\]

where \(\zeta^{j \to j'}(x) = c \log(\phi^j_0(x))\) is the prior
attractiveness of moving from location \(j\) to location \(\hat{j}\).
Note that this have the implication: - When an option seems very
attractive a priori, then it has a relatively high probability of being
selected even if its true value is low. (verbatim quote from
(\protect\hyperlink{ref-matejkaRationalInattentionDiscrete2015}{Matějka
and McKay 2015})).

I choose to have a uniform prior to simplify the exposition and because
I think it is a reasonable assumption. But I also can see how relaxing
this assumption could have interesting implications for the model.

We already stablished that with costless information the limit behavior
is that the worker perfectly directs their search to the highest value
location. Next, note that
\(\lim_{c \to \infty} \exp(u^{j\to \hat{j}}(x) / c) = 1\) thus the limit
behavior as the cost of information rises is dissregard new information
decide based on the prior

\begin{itemize}
\tightlist
\item
  With uniform prior:
  \(\lim_{c \to \infty} \phi^j(x,\hat{j}) = \frac{1}{\mid \mathcal{J} \mid}\)
  which is random search.
\item
  With some other prior:
  \(\lim_{c \to \infty} \phi^j(x,\hat{j}) = \phi^j_0(x)\).
\end{itemize}

Here the scaling parameter \(c\) plays a similar role to the noise
parameter in the discrete choice model from
(\protect\hyperlink{ref-lentzCompetitiveRandomSearch}{Lentz and Moen,
n.d.}).

\hypertarget{unemployed-worker}{%
\subsubsection{Unemployed Worker}\label{unemployed-worker}}

Unemployed workers receive instant utility from living in location
\(j\), \(b(x,j)\), and anticipate the probability of getting an offer
\(p^j\) in each location. They will choose the strategy that maximizes
their future expected value knowing that in each location they will
receive an offer which can be from any firm with a likelihood
proportional to the share of total vacancies posted by each firm in each
market. The worker will accept only the offers that promise her a higher
value than unemployment:

\begin{align*}
U^{j}(x) = \underbrace{b(x,j)}_{\text{instant utility}} + &\beta\max_{\phi^j_u(x)}\left\{ \underbrace{-c(\phi^j_u(x))}_{\text{cost of search strategy}} \right.  + \\
& \sum_{j'\in\{1,2\}} \underbrace{\phi^j_u(x, j')}_{\text{weight by probability of search in} j'}\left[ \overbrace{(1-p^{j'})U^{j}(x)}^{\text{no offer, stays unemployed now in} j'} \right.  \\
& \left. \left. \hspace{0cm} + p^{j'} \underbrace{\int \max\left\{U^{j}(x),W^{j\to j'}_{0}(x,y)\right\}\frac{v^{j'}(y)}{V^{j'}}dy}_{\text{if offer, pays cost,moves to } j \text{ and then is matched randomly with some firm}} \right] \right\} 
\end{align*}

I use the same bargaining as in
(\protect\hyperlink{ref-cahucWageBargainingOntheJob2006}{Cahuc,
Postel-Vinay, and Robin 2006}) and
(\protect\hyperlink{ref-baggerEmpiricalModelWage2019a}{Bagger and Lentz
2019}). The Nash bargaining solution implies that the worker recieves a
constant share \(\xi\) of the match rent, where \(\xi\) is the
bargaining power of the worker:

\[W^{j \to j'}_{0}(x,y) = U^{j}(x) + \xi S^{j\to j'}(x,y) =  (1 - \xi) U^{j}(x) + \xi [J^{j'}(x,y) - F^{j \to j'}]\]

From \eqref{eq:optimal-search-strategy} we know that the optimal
strategy for each unemployed worker is:

\begin{equation}\label{eq-optimal-search-unemployed}
\phi_{u}^{j}(x,j')=\frac{\exp{\left(\left[  \xi \int \max\left\{0, S^{j\to j'}(x,y) \right\} d\Gamma^{j'}(y) \right] / c\right)}}{\sum_{j'\in\{1,2\}}\exp{\left(\left[\xi \int \max\left\{0, S^{j\to \tilde{j} }(x,y) \right\} d\Gamma^{\tilde{j}}(y) \right] / c\right)}}
\end{equation}

Substituting the optimal strategy in the Bellman equation we get:

\begin{equation}\label{eq-bellman-unemployed}
U^{j}(x) = b(x,j) + \beta\left[U^j(x)  + c\log\left(\frac{1}{2}{\sum_{j'\in\{1,2\}}\exp{\left(\left[ \xi \int \max\left\{0, S^{j\to j' }(x,y) \right\} d\Gamma^{j'}(y) \right] / c\right)}} \right)\right]  
\end{equation}

\hypertarget{joint-value-of-a-match}{%
\subsubsection{Joint Value of a Match}\label{joint-value-of-a-match}}

\begin{itemize}
\tightlist
\item
  If a match between a worker and a firm in location \(j\) is destroyed
  the firm will get \(0\) and the worker gets their unemployment value
  in that location \(U^{j}(x)\).
\item
  Matches are destroyed for two reasons:

  \begin{itemize}
  \tightlist
  \item
    \emph{Exogenous destruction} with probability \(\delta\)
  \item
    \emph{Endogenous destruction}, if and only if \(J^j(x,y) < U^j(x)\).
  \item
    Denote
    \(\lambda^j(x,y) = (1-\delta)\mathbb{1}_{\{J^j(x,y)>U^{j\to j}(x,y)\}}\)
    the probability that a match survives accounting for both exogenous
    and endogenous destruction.
  \end{itemize}
\end{itemize}

We can write the Bellman equation of a match value as:

\begin{align*}
J^j(x,y) = \underbrace{f(x,y,j)}_{\text{match value added}} &+ \beta\left[  \overbrace{ (1-\lambda^j(x,y)) }^{\text{match is destroyed}}\underbrace{U^{j}(x)}_{\text{worker gets unemployment value}} \right. + \\
  & \underbrace{(\lambda^j(x,y)}_{\text{match survives}} \max_{\phi_s^j(x,y)}\left\{-c(\phi_s^j(x,y)) +  \sum_{j'\in\{1,2\}}\phi^j_s(x,y,j')\left[ \overbrace{(1-sp^{j'})}^{\text{no new offers}} \underbrace{J^j(x,y)}_{\text{stays with same firm}} \right. \right. +\\
& sp^{j'} \left. \underbrace{\int\max\{J^{j}(x,y),W^{j\to j'}_{1}(x,y',y)\}\frac{v^{j'}(x)}{V^{j'}}dy'}_{\text{worker only accepts new offers if value is greater than current match}}  \right]
\end{align*}

When a type \(x\) worker employed at a type \(y\) firm in location \(j\)
meet potencial type \(y'\) poaching firm at location \(j'\) (assume that
\(y'>y\) is productive enough to poach the woker, I'll show later what
the poaching condition is), competition between the two employers over
the worker's services occurs as a second price auction. No employer will
offer more that the entire match value \(J^j(x,y)\), thus this becomes
the reservation value for the worker. Nash bargaining solution implies
that the worker will obtain their outside option plus a share \(\xi\) of
the match surplus.

Bargaining implies that a worker employed in location \(j\) by firm
\(y\) when matched with firm \(y'\) in location \(j'\) gets:
\[W^{j \to j'}_{1}(x,y \to y') = J^{j}(x,y) + \xi [J^{j'}(x,y') - J^{j}(x,y) - F^{j' \to j}]\]

Note that this means that to be able to poach from different locations
the firm must be at least \(F^{j\to j'}\) more productive. In terms of
joint surplus we can rewrite
\[W^{j \to j'}_{1}(x,y \to y') = J^{j'}(x,y') + \xi \left[S^{j\to j'}(x,y') - S^{j \to j(x,y)}\right]\]

Focussing on the maximization problem in the Bellman equation we can
write:

\begin{align*}
J^j(x,y) + \max_{\phi_s^j(x,y)} \left\{  \sum_{j'\in\{1,2\}}\phi^j_s(x,y,j') \left[ sp^j\int\max\{0,  \xi \left[S^{j\to j'}(x,y') - S^{j \to j(x,y)}\right] \}d\Gamma^{j'}(y')\right] \right\}
\end{align*}

From \eqref{eq:optimal-search-strategy} we know that the optimal
strategy for each employed worker is:

\[\phi_{s}^{j}(x,y,j')=\frac{\exp{\left(\left[p^{j'} \xi \int \max\left\{0, S^{j\to j'}(x,y') - S^{j \to j}(x,y) \right\}d\Gamma^{j'}(y') \right] / c\right)}}{\sum_{\tilde{j} \in\{1,2\}}\exp{\left(\left[p^{\tilde{j}} \xi \int \max\left\{0, S^{j\to \tilde{j} }(x,y')-S^{j \to j}(x,y) \right\}d\Gamma^{\tilde{j}}(y') \right] / c\right)}}\]

Plug back into the value function:

\begin{align*} \label{eq-bellman-match}
J^j(x,y) = f(x,y,j) & + \beta\left [ \left[  (1-\lambda^j(x,y))U^{j}(x)   +\lambda^j(x,y)J^j(x,y) \right] \right. \\ & \left. +\log\left(\frac{1}{2}\sum_{j'\in\{1,2\}}\exp{\left(\left[p^{j'} \xi \int \max\left\{0, S^{j\to j' }(x,y')-S^{j \to j}(x,y) \right\}d\Gamma^{j'}(y') \right] / c\right)} \right) \right]
\end{align*}

\hypertarget{match-surplus}{%
\subsubsection{Match Surplus}\label{match-surplus}}

Next I characterize the surplus function. First, define the instant
surplus of a match between a type \(x\) worker in location \(j\) and a
type \(y\) firm in location \(j'\) as:
\[s(x,y,j \to j) = f(x,y,j,z) - b(x,j')\]

We can compute the surpluss of matches in the same location as:

\begin{align}
S^{j \to j}(x,y) &= J^j(x, y) - U^{j}(x) \nonumber \\
& = s(x,y,j \to j) + \beta \lambda^j(x,y)\underbrace{\left[J^j(x,y) - U^j(x)\right]}_{S^{j\to j}(x,y)} + \beta c \left[ \Lambda^j_1(x, y) - \Lambda^j_0(x)\right]  \nonumber\\
& = s(x,y,j \to j) + \beta (1-\delta)\max\{0, S^{j\to j}(x,y)\} + \beta c \left[ \Lambda^j_1(x, y) - \Lambda^j_0(x)\right] \label{eq-surplus-same-location}
\end{align}

Notice that \(\lambda^{j}(x,y) > 0\) if and only if
\(S^{j\to j}(x,y) > 0\), thus
\[(1-\delta)\max\{0, S^{j\to j}(x,y)\} =\lambda^{j}(x,y) S^{j\to j}(x,y) \]

The terms \(\Lambda_0(x)\) and \(\Lambda_1(x,y)\) come from the Bellmans
of the worker and the match respectively and are defined as:

\begin{align*}
\Lambda_1(x, y) &= \log \left(\sum_{j'\in\{1,2\}}\exp{\left(\left[\xi \int \max\left\{0, S^{j\to j' }(x,y')-S^{j \to j}(x,y) \right\}\Gamma^{j'}(y') \right] / c\right)} \right)  \\
\Lambda_0(x) &= \log \left(\sum_{j'\in\{1,2\}}\exp{\left(\left[\xi \int \max\left\{0, S^{j\to j' }(x,y') \right\}d\Gamma^{j'}(y') \right] / c\right)}\right)
\end{align*}

Finally we can use the expression in \eqref{eq-surplus-same-location} to
write the surplus of a match between a type \(x\) worker in location
\(j\) and a type \(y\) firm in location \(j'\) as:

\begin{align}
S^{j \to j'}(x,y) &= J^{j'}(x, y) - \left[U^{j}(x) + F^{j \to j'}\right] \nonumber \\
& = [J^{j'}(x, y) - U^{j'}(x)] + U^{j'}(x) - \left[U^{j}(x) + F^{j \to j'}\right] \nonumber \\
& = S^{j'\to j'}(x,y) - [U^{j}(x) - U^{j'}(x) + F^{j \to j'}] \label{eq-surplus}
\end{align}

From the expressions in \eqref{eq-surplus-same-location} and
\eqref{eq-surplus} the following is evident:

\begin{itemize}
\tightlist
\item
  A worker in location \(j\) can be hired from a firm in location \(j'\)
  by firm \(y\) if and only if the surplus of the match is positive:
  \[S^{j\to j'}(x,y) \geq 0 \qquad \iff \quad = J^{j'}(x,y) - U^{j\to j'}(x) \geq 0\]
\item
  A worker employed at location \(j\) by firm \(y\) can be poached by a
  firm \(y'\) in location \(j'\) if and only if the surplus of the match
  is higher than the surplus of staying at the same firm in the same
  location:
\end{itemize}

\begin{align*}
S^{j\to j'}(x,y') > S^{j\to j}(x,y) &\iff J^{j'}(x,y') - U^{j\to j'}(x) > J^{j}(x,y) - U^{j\to j}(x) \\
&\iff J^{j'}(x,y') - [U^{j}(x) - F^{j\to j'}] > J^{j}(x,y) - U^{j}(x) \\
&\iff J^{j'}(x,y') > J^{j}(x,y) +  F^{j\to j'}
\end{align*}

\hypertarget{vacancy-creation}{%
\subsubsection{Vacancy Creation}\label{vacancy-creation}}

\begin{itemize}
\tightlist
\item
  \(B^j(y)\) is the expected value of a type \(y\) vacancy making
  contact with a worker in location \(j\). Vacancies meet unemployed and
  employed type-\(x\) workers at a rates
  \[\frac{u_{+}^j(x)}{L^j} \qquad \text{and} \qquad s\frac{h_{+}^j(x,y)}{L^j}\]
  The expected value of posting a vacancy is therefore, the surplus that
  the posting firm expects to add, potential matches with negative
  surplus are immediately destroyed therefore those add no surplus. In
  terms of the Bellman equation we can write:
\end{itemize}

\begin{align}
B^j(y) &= \underbrace{\sum_{j'\in\{1,2\}}\left( \int\phi_u^{j'}(x,j)\underbrace{\frac{u_{+}^{j'}(x)}{L^{j}}}_{\text{likelihhod of match}}\times\overbrace{ \max\{0, S^{j'\to j}(x,y)\} }^{\text{match survives}}dx\right)}_{\text{expected value added from hiring unemployed workers}} \nonumber \\
& +\underbrace{\sum_{j'\in\{1,2\}}\left(\int \left(\int\underbrace{s\phi_s^{j'}(x,y',j)\frac{h_{+}^{j'}(x,y)}{L^{j}}}_{\text{likelihood of match}}\times \overbrace{ \max\{0, S^{j' \to j}(x,y)-S^{j' \to j'}(x,y')\}}^{\text{poaching is succesfull}}dx\right)dy'\right)}_{\text{expected value added from poaching other firms employees}} \label{eq-vacancy-posting}
\end{align}

Firms will post vacancies such that the marginal cost of the vacancies
and the marginal expected benefit \(B^j\) are equal:

\begin{equation}
  \label{eq-vacancy-posting-clearing}
  c_{j}'(v^{j}(y))=q^{j}B^{j}(y)
\end{equation}

Since the cost function is increasing and concave there is a unique
vacancy posting level that clears the market.

\hypertarget{labor-market-flows}{%
\subsection{Labor Market Flows}\label{labor-market-flows}}

Now we characterize the flows of workers in-to and out-of unemployment
at each location :

\begin{itemize}
\item
  I start by denoting the following indicator functions
  \[\eta^{j' \to j}(x,y) = \mathbb{1}_{\{S^{j' \to j}(x,y)>0\}} \qquad \eta^{j' \to j}(x,y'\to y) = \mathbb{1}_{\{S^{j' \to j}(x,y) > S^{j' \to j'}(x,y')\}}\]
  note that these characterize (if and only if) when a worker can hired
  from unemployment or to be poached from another firm.
\item
  Interactign the indicators with workers strategies we can obtain the
  actual probability that a woker moves across locations and firms:
\end{itemize}

\begin{equation}
\hat{\phi}^{j\to j'}_{u}(x,y) = \phi^{j}_{u}(x,j')\eta^{j' \to j}(x,y) \qquad \hat{\phi}^{j \to j'}_{s}(x,y\to y') = \phi^{j}_{s}(x,y,j')\eta^{j' \to j}(x,y\to y')
\end{equation}

We can write the \emph{interim} distributions in terms of theese
indicators as:

\begin{align}
u^{j}_{+}(x) &= u^j(x) + \int\left(1 - (1 - \delta)\eta^{j\to j}(x,y) \right)h^j(x,y)dy \label{eq-interim-u} \\
h_{+}^{j}(x,y) &= (1-\delta)\eta^{j \to j}(x,y)h^j(x,y) \label{eq-interim-h}
\end{align}

I split the mass of employed workers into three components:

\begin{equation}
\label{eq-law-of-motion-emp}
h^j(x,y) =  h^j_{U}(x,y) + h^j_{P}(x,y) + h^j_{R}(x,y) 
\end{equation}

those workers that are hired from unemployment, those that are poached
from other firms and those that the firm is able to retain.

\begin{itemize}
\tightlist
\item
  The mass of workers hired from unemployment:
\end{itemize}

\begin{equation}
h^j_{U}(x,y) = \gamma^{j}(y)\left[u_{+}^{1}(x)\hat{\phi}_{u}^{1 \to j}(x,y) + u_{+}^{2}(x)\hat{\phi}_{u}^{2 \to j}(x,y) \right]
\end{equation}

\begin{itemize}
\tightlist
\item
  The mass of workers that are succesfully poached from other firms:
\end{itemize}

\begin{equation}
h^j_{P}(x,y) = s\gamma^{j}(y)\left[\int{ h_{+}^{1}(x,y') \hat{\phi}_{s}^{1 \to j}(x,y'\to y) dy'} + \int{ h_{+}^{2}(x,y') \hat{\phi}_{s}^{2 \to j}(x,y'\to y) dy'} \right]
\end{equation}

\begin{itemize}
\tightlist
\item
  The mass of workers that the firm loses to other firms:
\end{itemize}

\begin{equation}
h^j_{R}(x,y) =  h_{+}^{j}(x,y)\left(1 - s \int \hat{\phi}_{s}^{j\to 1}(x,y\to y')d\Gamma^{1}(y')\right) \left(1 - s \int \hat{\phi}_{s}^{j\to 2}(x,y\to y')d\Gamma^{2}(y')\right)
\end{equation}

Note that \(h^{j}(x,y) - h^{j}_{R}(x,y)\) substituting the expression
for \(h_{+}^j(x,y)\) from \eqref{eq-interim-h}:

\begin{equation}
h(x,y)\left[1 - (1-\delta)\eta^{j \to j}(x,y) \prod_{j' \in \{1,2\}}\left(1 - s \int \hat{\phi}_{s}^{j\to j'}(x,y\to y')d\Gamma^{j'}(y')\right) \right]
\end{equation}

And the RHS in \eqref{eq-law-of-motion-emp}:

\begin{equation}
 \gamma^{j}(y) \left(\sum_{j'\in\{1,2\}} \left( u_{+}^{j'}(x)\hat{\phi}_{u}^{j'\to j}(x,y) + s\int{ h_{+}^{j'}(x,y') \hat{\phi}_{s}^{j' \to j}(x,y'\to y) dy'} \right)\right)
\end{equation}

This give us the following expression for the law of motion of the
employment distribution:

\begin{equation}
\label{eq-law-of-motion-emp-long}
h^{j}(x,y) = \frac{\gamma^{j}(y) \left(\sum_{j'\in\{1,2\}} \left( u_{+}^{j'}(x)\hat{\phi}_{u}^{j'\to j}(x,y) + s\int{ h_{+}^{j'}(x,y') \hat{\phi}_{s}^{j' \to j}(x,y'\to y) dy'} \right)\right)}{
  1 - (1-\delta)\eta^{j \to j}(x,y) \prod_{j' \in \{1,2\}}\left(1 - s \int \hat{\phi}_{s}^{j\to j'}(x,y\to y')d\Gamma^{j'}(y')\right) 
}
\end{equation}

The distribution of unemployed workers is determined by the existing
distribution of unemployed at ``interim'' an the probability that they
are not hired by any firm in any location:

\begin{align}
u^j(x) &= u_{+}(x) \left(1 - \prod_{j\in \mathcal{J}}  \int \hat{\phi}_{u}^{j\to j'}(x,y) d\Gamma(y')\right) \nonumber\\
&= \left[u^j(x) + \int\left(1 - (1 - \delta)\eta^{j\to j}(x,y) \right)h^j(x,y)dy  \right]\left(1 - \prod_{j\in \mathcal{J}}  \int \hat{\phi}_{u}^{j\to j'}(x,y) d\Gamma(y')\right) \nonumber\\
&\implies u^j(x) = \left[\int\left(1 - (1 - \delta)\eta^{j\to j}(x,y) \right)h^j(x,y)dy  \right] \frac{\left(1 - \prod_{j\in \mathcal{J}}  \int \hat{\phi}_{u}^{j\to j'}(x,y) d\Gamma(y')\right) }{\prod_{j\in \mathcal{J}}  \int \hat{\phi}_{u}^{j\to j'}(x,y) d\Gamma(y')} \label{eq-law-of-motion-unemp-long}
\end{align}

Note that \eqref{eq-law-of-motion-unemp-long} is a standard steady-state
condition for unemployment, or Beveridge curve. Here the flow out of
unemployment equals the flow into unemployment in every location at
every skill level.

Then I can compute the distribution of skill in each location as:
\[\ell^j(x) = u^j(x) + \int h^j(x,y)dy\] and the total population in
each location as: \[\mu^j = \int \ell^j(x)dx\]

\hypertarget{bibliography}{%
\section*{References}\label{bibliography}}
\addcontentsline{toc}{section}{References}

\hypertarget{refs}{}
\begin{CSLReferences}{1}{0}
\leavevmode\vadjust pre{\hypertarget{ref-baggerEmpiricalModelWage2019a}{}}%
Bagger, Jesper, and Rasmus Lentz. 2019. {``An {Empirical Model} of {Wage
Dispersion} with {Sorting}.''} \emph{The Review of Economic Studies} 86
(1): 153--90. \url{https://doi.org/10.1093/restud/rdy022}.

\leavevmode\vadjust pre{\hypertarget{ref-cahucWageBargainingOntheJob2006}{}}%
Cahuc, Pierre, Fabien Postel-Vinay, and Jean-Marc Robin. 2006. {``Wage
{Bargaining} with {On-the-Job Search}: {Theory} and {Evidence}.''}
\emph{Econometrica} 74 (2): 323--64.
\url{https://doi.org/10.1111/j.1468-0262.2006.00665.x}.

\leavevmode\vadjust pre{\hypertarget{ref-cheremukhinTargetedSearchMatching2020}{}}%
Cheremukhin, Anton, Paulina Restrepo-Echavarria, and Antonella Tutino.
2020. {``Targeted Search in Matching Markets.''} \emph{Journal of
Economic Theory} 185 (January): 104956.
\url{https://doi.org/10.1016/j.jet.2019.104956}.

\leavevmode\vadjust pre{\hypertarget{ref-lentzCompetitiveRandomSearch}{}}%
Lentz, Rasmus, and Espen Rasmus Moen. n.d. {``Competitive or {Random
Search}?''}

\leavevmode\vadjust pre{\hypertarget{ref-liseMacrodynamicsSortingWorkers2017}{}}%
Lise, Jeremy, and Jean-Marc Robin. 2017. {``The {Macrodynamics} of
{Sorting} Between {Workers} and {Firms}.''} \emph{American Economic
Review} 107 (4): 1104--35. \url{https://doi.org/10.1257/aer.20131118}.

\leavevmode\vadjust pre{\hypertarget{ref-matejkaRationalInattentionDiscrete2015}{}}%
Matějka, Filip, and Alisdair McKay. 2015. {``Rational {Inattention} to
{Discrete Choices}: {A New Foundation} for the {Multinomial Logit
Model}.''} \emph{American Economic Review} 105 (1): 272--98.
\url{https://doi.org/10.1257/aer.20130047}.

\leavevmode\vadjust pre{\hypertarget{ref-simsRationalInattentionLinearQuadratic2006}{}}%
Sims, Christopher A. 2006. {``Rational {Inattention}: {Beyond} the
{Linear-Quadratic Case}.''} \emph{American Economic Review} 96 (2):
158--63. \url{https://doi.org/10.1257/000282806777212431}.

\leavevmode\vadjust pre{\hypertarget{ref-simsImplicationsRationalInattention2003}{}}%
Sims, Christopher A. 2003. {``Implications of Rational Inattention.''}
\emph{Journal of Monetary Economics} 50 (3): 665--90.
\url{https://doi.org/10.1016/S0304-3932(03)00029-1}.

\leavevmode\vadjust pre{\hypertarget{ref-wuPartiallyDirectedSearch2020}{}}%
Wu, Liangjie. 2020. {``Partially {Directed Search} in the {Labor
Market}.''} PhD thesis, University of Chicago.

\end{CSLReferences}



\end{document}
